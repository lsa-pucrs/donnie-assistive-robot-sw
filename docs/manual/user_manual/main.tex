\documentclass[11pt,a4paper,final]{report}

% required for portuguese
%\usepackage[latin1]{inputenc}
%\usepackage[utf8]{inputenc}
%\usepackage[brazil]{babel} 
\usepackage{fancyvrb}
\usepackage[dvipsnames]{xcolor}
%\usepackage{rotating}
% C code
\usepackage{listings}
% %\lstset{language=C, tabsize=4, basicstyle=\ttfamily}
\usepackage{color}
% link creation
\usepackage[bookmarks,pdftitle={Donnie User Manual}, pdfauthor={PUCRS/LSA - 2016}]{hyperref}
% para fazer tabelas bonitas
\usepackage{booktabs}
\usepackage{subfigure}
\usepackage{rotating}
\usepackage{pdflscape}
\usepackage{longtable}
\usepackage{multirow}
%\usepackage{xcolor}
%\usepackage[table]{xcolor}
\usepackage{colortbl}
\usepackage{placeins}
\usepackage{pbox}
% C code
\usepackage{listings}
% Tikz drawings
\usepackage{tikz}
\usetikzlibrary{arrows, automata, calc, positioning}

\lstset{language=C, tabsize=4, basicstyle=\ttfamily, commentstyle=\color{cyan}, stringstyle=\color{blue}, 
keywordstyle=\color{magenta}, morekeywords= {uint16\_t, uint32\_t, size\_t, class}}
% adjust size of margins
\usepackage[left=2cm, right=2cm]{geometry}
\geometry{margin=2cm}

\author{PUCRS/LSA - 2016}
\title{Donnie User Manual}
\date{\today\\v0.1}

% TC counter
\newcounter{tcnum}
\setcounter{tcnum}{0}

% TODO notes
\usepackage[colorinlistoftodos,prependcaption,textsize=tiny]{todonotes}


\begin{document}

\pagestyle{empty}

\maketitle 

\tableofcontents

\chapter{Introduction}

\section{Introduction}
\label{sec:intro}

The {\bf goal} of this manual is to help new software designers interested to contribute and improve Donnie's software environment. There are several different ways to contribute, such as:

\begin{itemize}
\item with bug reports to the github website;
\item improving the documents or with translation;
\item adding new assistive technology (text-to-speech, screen reader, magnifier, etc);
\item extending Donnie to new programming languages (C, Python, Java, etc);
\item adding new software to make Donnie smarter (image processing, voice commands, improved localization, etc);
\end{itemize}

\section{Software Resources}

Donnie's software is built on top of robotic middlewares. We initially support \href{http://playerstage.sourceforge.net/}{Player} middleware since it is lightweight for running on low cost embedded hardware, but it is not limited to it. We plan to support \href{http://www.ros.org/}{ROS} in the near future.

Please refer to \href{https://github.com/jennyhasahat/Player-Stage-Manual}{Jenny's Player/Stage tutorial} to learn how to use Player/Stage. Although Jenny's tutorial is a great introduction to Player/Stage, you wont find a tutorial about how to develop drivers and interfaces for Player. We plan to build such manual .... one day !!!

Donnie also uses several libraries that are detailed in the appropriate middleware section.


\section{Issue Report}

Donnie uses \href{https://github.com/lsa-pucrs/donnie-assistive-robot-sw/issues}{Github Issue List} to track features requests, bugs, etc.


\chapter{Setup}

\section{Donnie's Use Cases}
\label{sec:user_case}

Donnie can be used in two modes: with a simulated robot and with the actual robot.
So, if you still don’t have your Donnie robot, you can start using the simulation environment.


\section{Simulating Donnie}
\label{sec:sim}

\subsection{Virtual Machine}

\subsection{Pre-Built Software}

\subsection{Compiling the Software}

\subsection{Launch the Simulation Environment}

There are some pre-built simulation environments to start testing Donnie.
Copy the following files to a local directory

DONNIE\_PATH/sssssss

Execute the command cccccc.


\section{Using the Actual Donnie Robot}
\label{sec:robot}

The setup to use the robot is to the setup for simulation, except for few additional steps described below.

\subsection{Virtual Machine}

\subsection{Pre-Built Software}

\subsection{Compiling the Software}


\chapter{GoDonnie}

\section{Introduction}
\label{sec:godonnie_intro}

This part of the manual is required only if the reader wants to learn the GoDonnie language. 

\subsection{Why to Build GoDonnie}

GoDonnie has been built to teach the basics of programming for people with normal sight, visual impairment, or blind.
GoDonnie is heavily inspired by Logo ....

{\ TODO}

Once the student dominates the GoDonnie language, he can migrate to more advanced languages such as Python, C, Java, etc. 
All these languages are also supported by the robotic middlewares (Player and ROS) so the transition of languages comes quite naturally, without changing the underlying Donnie's software stack. 

\subsection{Audience to Use GoDonnie}

\subsection{Features}




\section{Grammar and Syntax}
\label{sec:godonnie_grammar}

Some me some code snapshots !!!

\section{GoDonnie Interpreter}
\label{sec:godonnie_interp}

How to use it !

\section{Example Codes}
\label{sec:godonnie_codes}


\subsection{My First Program}

Donnie says somthing nice

\subsection{My Second Program}

Donnie walks back and forth.

\subsection{Square Program}

Donnie is alert, patrolling in a square shape.

\subsection{Vibration Program}

Donnie is vibrating.

\chapter{Custom Environments}

\section{Custom Environments}
\label{sec:envs}

The user can extend both the simulated and the actual robot environments.

\subsection{Building New Virtual Environments}

New simulation environments can be easily created with the help of \href{http://playerstage.sourceforge.net/?src=stage}{Stage}.
Please refer to \href{https://github.com/jennyhasahat/Player-Stage-Manual}{Jenny's Player/Stage tutorial} to learn how to create a robot and an environment with Player/Stage. 

\subsection{Building a New Real Environment}

We recommend to use Donnie robot in controlled environment, specially because of the its computer vision 
features which can have poor performance in a uncontrolled environment.
For this reason we recomment maze-like envionments with opaque collors (with low reflex) to work with Donnie.

We provided STL files of modular wall designs for the ones with access to Laser Cutting facility.
These walls have 40 cm of height, to block Donnie's view from the uncontrolled environment.
These modular walls can be used to build envirinments like the ones illustrated in Figure~\ref{fig:maze}.


 


\begin{landscape}
%##############################
%##############################
\chapter{History}
\label{sec:history}
%##############################
%##############################

% remove page number
\thispagestyle{empty}

\begin{tabular}{p{1.5cm}p{3cm}p{12cm}p{6cm}}
	\toprule
	 \textbf{Version} & \textbf{Date} & \textbf{Description}  & \textbf{Authors}\\
	\midrule
v0.2 & Sept/20/2016 & some nice comment here & Fulano da Silva and Ciclano de Souza  \\
v0.1 & Sept/19/2016 & initial document structure is created & Alexandre Amory  \\
	\bottomrule
\end{tabular}

\end{landscape}

\end{document}
