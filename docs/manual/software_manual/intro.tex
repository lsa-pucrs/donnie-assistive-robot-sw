\chapter{Introduction}

\section{Introduction}
\label{sec:intro}

The {\bf goal} of this manual is to help new software designers interested to contribute and improve Donnie's software environment. There are several different ways to contribute, such as:

\begin{itemize}
\item with bug reports to the github website;
\item improving the documents or with translation;
\item adding new assistive technology (text-to-speech, screen reader, magnifier, etc);
\item extending Donnie to new programming languages (C, Python, Java, etc);
\item adding new software to make Donnie smarter (image processing, voice commands, improved localization, etc);
\end{itemize}

\section{Software Resources}

Donnie's software is built on top of robotic middlewares. We initially support \href{http://playerstage.sourceforge.net/}{Player} middleware since it is lightweight for running on low cost embedded hardware, but it is not limited to it. We plan to support \href{http://www.ros.org/}{ROS} in the near future.

Please refer to \href{https://github.com/jennyhasahat/Player-Stage-Manual}{Jenny's Player/Stage tutorial} to learn how to use Player/Stage. Although Jenny's tutorial is a great introduction to Player/Stage, you wont find a tutorial about how to develop drivers and interfaces for Player. We plan to build such manual .... one day !!!

Donnie also uses several libraries that are detailed in the appropriate middleware section.


\section{Issue Report}

Donnie uses \href{https://github.com/lsa-pucrs/donnie-assistive-robot-sw/issues}{Github Issue List} to track features requests, bugs, etc.

