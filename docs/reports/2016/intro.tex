\section{Introduction}
\label{sec:intro}

%Para pessoas que possuem deficiência visual (DV), as habilidades de O & M são indispensáveis para a conquista da autonomia e, consequentemente, para sua independência e inclusão na escola e na sociedade \cite{dutra2003}. Existem programas específicos para desenvolver e treinar O & M, tais como da American Foundation for the Blind - http://www.afb.org/,  Association for Education and Rehabilitation of the blind and visually impaired - http://aerbvi.org/, National Center on Deaf-Blindness - https://nationaldb.org/ ...). Alguns programas estão direcionados a professores e especialistas em O e M enquanto outros estão direcionados a pessoas com DV. O trabalho aqui apresentado serve como ferramenta para o desenvolvimento ou treinamento das habilidades de O & M por meio da prática do pensamento computacional. Para tal, faz uso de um robô que deve ser deslocado em um cenário por meio de programação.
The skills of orientation and mobility (O\&M) are indispensable for visually impaired people  because they increase their autonomy, their independence, and inclusion in school and society \cite{dutra2003}. 
%A visually impaired person is {\bf colocar uma definicao aqui}\cite{}. 
There are specific programs to develop and to train O\&M, such as the American Foundation for the Blind\footnote{\url{http://www.afb.org/}}, Association for Education and Rehabilitation of the blind and visually impaired\footnote{\url{http://aerbvi.org/}} and National Center on Deaf-Blindness\footnote{\url{https://nationaldb.org/}}. Some programs are directed to teachers and experts in O\&M, while others are directed to visually impaired people. 

%O uso da robótica possui caráter multidisciplinar e requer conhecimentos das áreas de matemática, física, mecânica, computação, etc., podendo ser um ferramenta  útil para o desenvolvimento e uso do pensamento computacional para resolução de desafios no campo da psicomotricidade e da orientação e mobilidade (O & M). Desta forma, a Robotica passa a ser um importante instrumento para o desenvolvimento de habilidades que necessitam de pensamento lógico e abstrato, podendo aproximar aspectos teóricos e complexos de ações que são práticas e lúdicas.
The use of robotics has multidisciplinary nature and requires knowledge of the areas of mathematics, physics, mechanics, computing, etc. Previous papers show that robotics programming can be a useful tool for the development of computational thinking for solving challenges in the field of motor skills and O\&M\cite{azevedo2010}. However, these results evaluated people with normal vision and, as demonstrated next, most studies use environments and tools that are not adequate for people with visual impairment.
%Thus, robotics programming can be an important tool for the development of skills that require logical and abstract thinking, and it may approximate theoretical and complex aspects of actions that are practical and ludic.
% A.M.A.: fala a mesma coisa q o frase acima. acho que ficou repetitivo


%Existem diversas linguagens de programação para fins educacionais. Por exemplo, Logo e suas variações, tais como NetLogo (\url{http://ccl.northwestern.edu/papers/agent2004.pdf}), Kturtle (\url{https://edu.kde.org/kturtle/}), StarLogo (), Micromundos (), Megalogo (), SLogo (), Etoys (\url{http://www.ufrgs.br/soft-livre-edu/software-educacional-livre-na-wikipedia/etoys-linguagem-de-programacao/}), dentre outros. Também há ferramentas de programação em blocos de comandos, tais como Scratch (\url{http://scratch.mit.edu/}),Gameblox (\url{http://education.mit.edu/portfolio_page/gameblox/}), por exemplo. De forma geral, essas linguagens e ferramentas de programação suportam uma aprendizagem construtivista suportada por uma linguagem de scripts. 
There are several programming languages for educational purposes. For example, Logo and its variations, such as 
NetLogo\footnote{\url{http://ccl.northwestern.edu/papers/agent2004.pdf}}, Kturtle\footnote{\url{https://edu.kde.org/kturtle/}}, StarLogo\footnote{\url{http://education.mit.edu/starlogo-tng/}} and
Etoys\footnote{\url{http://www.ufrgs.br/soft-livre-edu/software-educacional-livre-na-wikipedia/etoys-linguagem-de-programacao/}}. There is also programming tools into command blocks, such as, for example, 
Scratch\footnote{\url{http://scratch.mit.edu/}} and Gameblox\footnote{\url{http://education.mit.edu/portfolio_page/gameblox/}}. 
In general these programming languages and tools support a constructivist learning backed by a scripting language. However, most of these programming environments are highly graphical and they do not provide an accessible interface for visually impaired people.

There are few attempts to use existing robotics programming environments for students with visually impairment \cite{ludi2011, howard2012}. Most of these prior works use Lego Mindstorms with some encouraging results. On the other hand, they also present some shortcomings and deficiencies in terms of usage. Most of these issues are caused because the programing environment was mainly built for people with normal vision and it was adapted for visually impaired people. Moreover, both papers \cite{ludi2011, howard2012} also do not evaluate the possible improvement on O\&M skills of the students. 

Thus, the \emph{goal} of this paper is to present a tool for the development or training of O\&M skills through the practice of computational thinking. 
%De acordo com Figueira (2000), a mobilidade das pessoas que são cegas depende de dois fatores, que são orientação mental e locomoção física. O trabalho aqui descrito pretende servir como um recurso ao desenvolvimento da habilidade das pessoas reconhecerem o espaço a sua volta e as relações espaciais em relação a si própria e a outros, que, neste caso, se estabelece através de um robô. Desta forma, a pessoa que é cega deverá formar um mapa mental do cenário para que o robô possa ser deslocado. Para tal, o sistema de referência será constituído de referências sonoras e táteis. O cenário deverá emitir sons para que a pessoa possa interpretar os diferentes sinais que são fornecidos pelo ambiente, e que devem lhe servir como pontos de referência. Assim, será possível saber se o deslocamento do robô está correto, e o que há ao redor dele. Os sentidos da orientação, concentração e memória por parte da pessoa que é cega deverão permitir que deslocar o robô com mais facilidade e domínio. 
According to \cite{figueira2000}, the mobility of persons who are blind depends on two factors, which are mental orientation and physical mobility. The proposed tool intends to serve as a resource to the development of people ability, to recognize the space around it and the spatial relationships in relation to itself and to others, which in this case is established by a robot. The user moves the robot in a tactile scenario through programming, externalizing the computational thinking, which includes logical reasoning for problem-solving actions \cite{zanetti-oliveira2015}. %A tactile model is used to represent the robot's scenario. 
In this way, the person who is blind must form a mental map of the scene so that the robot can be moved. For this, the reference system shall consist of sound and tactile references. The scenario and the robot should emit sounds so the person can interpret the different signals that are provided by the environment, serving as reference points. Thus, the user can know if the robot movements are correct and can learn the environment around it. The user's senses of orientation, concentration and memory should allow the robot to move more easily in the scenario.
%on an external medium, which may have already been internalized or what might be being built or tested {\bf nao entendi esta frase. nao eh o robot q tem concentracao e orientacao .... corrigi no português. Rever em inglês.... Marcia}.
Our \emph{assumption} is that, by using the proposed tool, the user with visually impairment improves his/her analytical, programming, and O\&M skills at the same time. Moreover, the ludic aspect of robotics might encourage, specially young students, to use the proposed tool and to have more practicing time.

%O robô será deslocado no cenário por meio de programação, externalizando o pensamento computacional, que inclui o raciocínio lógico até o comportamento humano para a ação de resolução de problemas (Zanetti e Oliveira, 2015). O deslocamento ocorrerá em um cenário, que funcionará como uma maquete. Essa poderá ser do tipo informacional ou educacional (Martins e Abreu, 2008). Na maquete informacional, as representações serão fixas, e possibilitará ao usuário o conhecimento do espaço físico que ela representa e a exploração desse. Na maquete educacional, os objetos estarão em constante modificação. Assim, além do seu caráter informativo, também permitirá que a pessoa cega possa expressar a sua compreensão do ambiente físico.

%This may be the informational or educational type \cite{martins-abreu2008}. In the informational tactile model, the representations are fixed, allowing the user to get knowledge of the physical space it is and exploring it. In the educational tactile model, objects are constantly changing. Thus, besides its informative nature, it also allows the blind person to express their understanding of the physical environment.
% para o termo maquete, eu não usaria mockup. Mas, sim, Tactile Model... Marcia
% A.M.A: me parece que esta parte comentada nao se encaixar direito com o restante do texto. me parece algo muito específico, não apropriado para a fase atual do projeto.


%No contexto desta pesquisa, foi elaborada uma linguagem de programação baseada em scripts para fins educacionais como uma ferramenta no auxílio de resolução de problemas espaciais, que envolvam orientação e mobilidade. Foi concebida sobre critérios de acessibilidade visando seu uso por pessoas com deficiência visual.
%In the context of this research, a programming language based on scripts for educational purposes was developed as a tool to aid in solving spatial problems involving orientation and mobility. It was designed for accessibility criteria aiming its use by visually impaired persons.
%A.M.A.: acho que este parágrafo está redundante

To the best of the author's knowledge, this is the first robotic programming environment that considers assistive and inclusive features for visual impaired people as the number one priority. The proposed system consists of a new educational/assistive robot called Donnie, a new assistive programming language called GoDonnie, and a software environment that encourages students to study beyond their initial programming skills. Both the hardware and the software architectures are completely based on open technology and they are freely available at Donnie's website \footnote{\url{https://github.com/lsa-pucrs/donnie-assistive-robot-sw}}. Donnie's hardware architecture is based on resources commonly used for educational resources such as Arduino, Raspberry Pi, and 3D printer parts. The software architecture is based on Logo programming language built on top of Player/Stage~\cite{gerkey2003} robotic framework running on Linux operating system. Both the hardware and the software were designed to be modular and expandable. 

The main unique contributions of this paper can be summarized as:
\begin{itemize}
\item the development of a Logo-based inclusive programing language called GoDonnie for visually impaired people;
\item integration of an educational programing language to Player framework;
\item the use of Logo-based language to teach programming for visually impaired people;
\item the use of robotics to develop O\&M skills in visually impaired people;
\item an extension of Stage simulation environment to include sound feedback and text-to-speech for visually impaired people;
\item improvements on the sound feedback and text-to-speech resources available at Player framework;
\item the first Arduino firmware compatible with Player robotic framework, including extensive documentation and manual;
\item lower hardware cost compared to the Lego Mindstorms.
\end{itemize}

This paper is organized as follows.  Section~\ref{sec:related} reviews prior papers about the use of robotics for people with visual disabilities with focus on initiatives to promote Computer Science as a future profession for those students. Section~\ref{sec:system} describes the entire system architecture (both hardware and software designs) and the main design decisions to promote a more universal interface the proposed system. Section~\ref{sec:usage} suggests some usage scenarios and future tests/evaluation that can be executed with the proposed system. Section~\ref{sec:conclusion} presents final remarks and future work.