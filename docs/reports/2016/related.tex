\section{Related Works}
\label{sec:related}

A literature review showed that accessible programming environments for visually impaired students have already been subjects of research~\cite{Kakehashi2013, Kakehashi2014}, specially using robots~\cite{ludi2011,howard2012,ludi2014,  dorsey2013}. There are several resources available nowadays to be used for this purpose, such as Lego Mindstorms NXT robots, Bricx Command Center (BricxCC) programming environment, Wiimote for tactile feedback, screen reader and Text Magnifier/Reader as JAWS and Zoomtext, respectively. Ludi et al.~\cite{ludi2011} describe the results of the robotics programming workshops, using these tools, developed for participants with different degrees of vision. 

Ludi et al.~\cite{ludi2014} developed a software called JBrick, for students who are visually impaired. It provides accessibility features for the Lego Mindstorms NXT and was tested with ten participants, which demonstrated that they were satisfied in using it. Howard et al.~\cite{howard2012} also used a Lego Mindstorms NXT robot kit that was integrated with JAWS screen reader and MAGIC magnifier. The results of a pilot study with nine participants with different degrees of visually impairment showed that the use of multimodes of interaction enabled these students to participate in traditional robot programming processes. Another authors used BricxCC as a programming interface for the Lego Mindstorms NXT that was also integrated with JAWS and MAGIC and tested with visually impaired students~\cite{dorsey2013}.

Another approach for teaching programming mobile robots to visually impaired persons is to use cubic blocks and RFID tags, allowing operation via tactile information~\cite{Kakehashi2013,Kakehashi2014}. Thus, students can control a mobile robot by positioning wooden blocks on a mat. This system called P-CUBE, was used to operate a robot composed by an Arduino UNO microcontroller board, a wireless microSD shield, a buzzer, two motors, a gearbox, and batteries. It was tested with four visually impaired persons and was improved with their feedbacks.

Tran et al.~\cite{Tran2007} proposed an audio programming tool to help visually impaired persons. It is based on text-to-speech technology and used for learn programming, but it was not designed for children or teenage students. There are other works involving programming education for visually impaired students~\cite{Franqueiro2006, kane2014, Konecki2015}, however they are not based in the robotics-based curriculum.

It is possible to see that Lego Mindstorms NXT is used in several works, mainly for its affordable cost, but usually associated with others tools to provide accessibility. However, accessibility will depend on the associated programming environment, and several of them have limitations that make it difficult to use by visually impaired people. P-CUBE's tactile approach is interesting, however, the programming commands are very limited.  The availability of a audio programming tool with voice commands helps in accessibility, but it does not address the use of robots and it needs to be tested with children and younger people without programming experience.

%Os estudos apresentados anteriormente foram resultados de pesquisa baseada na técnica de Snowballing. Além dessa, foi realizada uma Revisão Sistemática da literatura que teve por objetivo identificar e compreender os procedimentos metodológicos que fizessem uso de robótica como apoio ao ensino de programação para pessoas que são cegas. Foram identificadas palavras chave relacionadas com o domínio-foco da revisão sistemática para incluir na string de busca. Foram utilizadas as bases ACM Digital Library, ScienceDirect, IEEExplore, Scopus. A string utilizada para procura de trabalhos nos mecanismos de busca foi \textit{TITLE-ABS-KEY (("blind" OR "visually impaired" OR "visual disability" OR "blindness" OR "student disability" OR "unsighted pupils" OR {visual impairments}) AND programm* AND Robot*)}. Essa string foi adaptada de acordo com o mecanismo de busca de cada base, porém de forma que não alterasse o seu sentido lógico.
These presented works were the result of a research review based on Snowballing technique. We also did a Systematic Review of the literature to identify and understand the methodological procedures related to the use of robotics as a tool to teach programming for people who are blind. The keywords related to the domain-focus have been identified to be included in the search string. The databases of the ACM Digital Library, ScienceDirect, IEEExplore and Scopus were used. The string used in search engines  was \textit{TITLE-ABS-KEY (("blind" OR "visually impaired" OR "visual disability" OR "blindness" OR "student disability" OR "unsighted pupils" OR {visual impairments}) AND programm* AND Robot*)}. This string has been adapted according to each search engine, however, without changing its Boolean sentence.

%Os critérios definidos para seleção dos estudos foram:
The criteria for selecting the papers were:
\begin{itemize}
%    \item Critérios de inclusão\\
%    I1- O resultado deve estar nos idiomas definidos (Inglês ou Português).\\
%    I2- O resultado deve estar disponível integralmente e ser disponibilizado pela instituição de ensino.\\
%    I3- O resultado deve conter no título, nas palavras-chave, ou no resumo alguma relação com o tema deste trabalho (usuário cego, programação, robótica).\\

%    \item Critérios de exclusão\\
%    E1- O resultado não está relacionado ao tema do trabalho.\\
%    E2- Resultados com idiomas diferentes do Inglês e do Português.\\
%    E3- No caso de resultados similares ou duplicados, somente o mais recente foi considerado.\\
%    E4- Resultados que abordem questões de ensino de programação de forma geral, e não com pessoas com deficiência visual.\\
%    E5- Resultados que abordem questões de uso de robótica de forma geral, e não com pessoas com deficiência visual.\\
%    E6- O resultado não ser da área de Ciência da Computação ou Engenharia.

    \item Inclusion criteria: \\
    I1- The result must be in the defined languages (English or Portuguese); \\
    I2- The result must be fully available for download; \\
    I3- The result must contain in the title, the keywords, or abstract any relation to the subject of this work (blind user, programming, robotics). \\

    \item Exclusion criteria: \\
    E1- The result is not related to the subject of this work; \\
    E2- The result is not in English or Portuguese languages; \\
    E3- In the case of similar or duplicated results, only the most recent is considered; \\
    E4- The result addresses programming education in general, but not with people with visual impairments; \\
    E5- The result addresses issues of robotics use in general, but  not with people with visually impairments; \\
    E6- The result is not from the areas of Computer Science or Engineering.
\end{itemize}


%A seleção dos artigos foi realizada da seguinte forma:
The selection of papers was performed as follows:
 \begin{enumerate}
%\item Execução da \textit{String} em cada base de busca.
%\item Aplicação de filtro nas bases para selecionar apenas trabalhos da área de Ciência da computação e Engenharia.
%\item Exportação da lista de trabalhos no formato \textit{bibtex}.
%\item Importação dos arquivos \textit{bibtex} na ferramenta \textit{Start}\footnote{$http://lapes.dc.ufscar.br/tools/start_tool$}, que é uma ferramenta que apoia na organização de revisões sistemáticas. 
%\item Foi realizado um primeiro filtro no qual foram aplicados os critérios de seleção com base na leitura do resumo, palavras-chave e título. Do total de 125 trabalhos foram aceitos 9 trabalhos. A Tabela \ref{Tabela I} apresenta a quantidade de artigos encontrados, duplicados e aceitos nas bases utilizadas.
%\item Após, foi realizado um segundo filtro, que confirmou os dados anteriores.

\item Execution of the \textit{String} in each database search engine;
\item Filter application in each database search engine to select only the works of Computer Science and Engineering areas;
\item Export the list of works in \textit{bibtex} format;
\item Import the \textit{bibtex} files in \textit{Start}\footnote{\url{http://lapes.dc.ufscar.br/tools/start_tool}}, which is a tool that supports the organization of systematic reviews;
\item Execution of a first filter in which the selection criteria were applied based on the reading of the abstract, keywords and title. Of the total of 125 papers just 9 were accepted. Table \ref{Tabela I} presents the number of papers found in the search, duplicated and selected;
\item Execution of a second filter that confirmed previous data.

 \end{enumerate}

\begin{table}[H]
\centering
%\caption{Artigos retornados na busca}
\caption{Papers returned by the search}
\label{Tabela I}
\begin{tabular}{|l|c|c|c|}
\hline
%\multicolumn{1}{|c|}{\textbf{Base}} & \textbf{\begin{tabular}[c]{@{}c@{}}Número de estudos \\ retornados\end{tabular}} & \textbf{\begin{tabular}[c]{@{}c@{}}Número de estudos\\ duplicados\end{tabular}} & \textbf{\begin{tabular}[c]{@{}c@{}}Número de artigos \\ selecionados\end{tabular}} \\ \hline

\multicolumn{1}{|c|}{\textbf{Base}} & \textbf{\begin{tabular}[c]{@{}c@{}}Returned  \\ papers\end{tabular}} & \textbf{\begin{tabular}[c]{@{}c@{}}Duplicated \\ papers\end{tabular}} & \textbf{\begin{tabular}[c]{@{}c@{}}Selected  \\ papers\end{tabular}} \\ \hline
Scopus                              & 66                                                                               & 3                                                                               & 8                                                                                  \\ \hline
ACM                                 & 19                                                                               & 4                                                                               & 0                                                                                  \\ \hline
IEEEexplore                         & 7                                                                                & 5                                                                               & 1                                                                                  \\ \hline
ScienceDirected                     & 33                                                                               & 0                                                                               & 0                                                                                  \\ \hline
\textbf{Total}                      & \textbf{125}                                                                     & \textbf{12}                                                                     & \textbf{9}                                                                         \\ \hline
\end{tabular}
\end{table}

%No que se refere ao resultado da revisão sistemática, destacam-se:
With regard the result of the systematic review we can highlight:

\begin{itemize}
    
\item The identified papers are from 2008 (1), 2009 (1), 2010 (1), 2011 (1), 2012 (1), 2013 (2), 2014 (2) and 2015 (1).

%\item Os procedimentos metodológicos incluiram a realização de oficinas (\cite{ludi2011,howard2012,ludi2014,Howard:2013,Kakehashi:2015, Ludi:2008}), uso de tutoriais (\cite{Ludi:2008,ludi2011,howard2012,Howard:2013}), técnicas de verbalização \cite{howard2012}.
\item The methodological procedures included the development of workshops (\cite{ludi2011,howard2012,ludi2014,Howard:2013,Kakehashi:2015, Ludi:2008}), the use of tutorials (\cite{Ludi:2008,ludi2011,howard2012,Howard:2013}), verbalization techniques \cite{howard2012}.

%\item Uso de maquetes para a movimentação do robô (\cite{Ludi:2008,Howard:2013,Kakehashi:2015}).
\item Use scale models for handling the robot (\cite{Ludi:2008,Howard:2013,Kakehashi:2015}).

%\item Realização de atividades em grupo (\cite{ludi2011, ludi2014,Ludi:2008,Howard:2013}) e atividades individuais (\cite{howard2012, Kakehashi:2015}).
\item Execution of group (\cite{ludi2011, ludi2014,Ludi:2008,Howard:2013}) and individual (\cite{howard2012, Kakehashi:2015}) activities.

%\item Uso do kit de robótica Lego Mindstorm NXT (\cite{Ludi:2008,ludi2010,ludi2011,howard2012, Howard:2013, ludi2014}) e uso de robô próprio com placa Arduino (\cite{Kakehashi2013,Kakehashi:2015,Motoyoshi:2015}).
\item Use of robotic kit Lego Mindstorms NXT (\cite{Ludi:2008,ludi2010,ludi2011,howard2012, Howard:2013, ludi2014}) and use of custom robot with Arduino board (\cite{Kakehashi2013,Kakehashi:2015,Motoyoshi:2015}).

%\item Uso de blocos lógicos de material concreto para representar comandos de programação para manipulação do robô (\cite{Kakehashi:2015,Kakehashi2013,Motoyoshi:2015}).
\item Use of logical blocks of hard material to represent the programming commands for moving the robot  (\cite{Kakehashi:2015,Kakehashi2013,Motoyoshi:2015}).

%\item Necessidade de integração do ambiente virtual com o robô de forma autônoma pelo usuário cego (\cite{Kakehashi2013,Motoyoshi:2015}).
\item The virtual environment and the physical robot must be integrated to improve independent and autonomous actions by the blind user (\cite{Kakehashi2013,Motoyoshi:2015}).

%\item Identificação de término de laços de programação,  \cite{Kakehashi:2015} e de melhor identificação de instruções dentro de outros comandos (por exemplo, as ações de If e Then) \cite{ludi2014}.
\item Identification of the end of programming blocks (including selection and loop blocks)~\cite{Kakehashi:2015} and better identification of instructions within other commands (for example, the actions of If and Then)~\cite{ludi2014}.

%\item Busca de compatibilidade da linguagem de programação com o software de leitura de tela (\cite{ludi2011,Ludi:2008}).
\item Test the programming language compatibility with the screen reader software (\cite{ludi2011,Ludi:2008}).

%\item Uso de interface multimodal, incluindo interface baseada em áudio e em vibração \cite{park2012}.
\item Use of multimodal interface, including interface based on audio and vibration \cite{park2012}.

%\item Dificuldade no uso do teclado para escrever chaves e colchetes \cite{ludi2014}.
\item Blind users have difficulty in using the keyboard to write brackets \cite{ludi2014}.

%\item Apoio de pessoas videntes para um melhor uso do ambiente de programação (\cite{Ludi:2008, ludi2011}).
\item Support of seers people to better use the programming environment \cite{Ludi:2008, ludi2011}.

\end{itemize}

%A revisão sistemática também permitiu identificar boas práticas no uso de ambiente de programação por pessoas que são cegas, bem como do uso de robôs por essas pessoas. Dentre essas, tem-se questões relacionadas ao uso de recursos e procedimentos de ensino, dinâmicas de trabalho com os alunos, formas de avaliação do aprendizado e do ambiente de programação, além de questões referentes ao local no qual ocorrerão as atividades e à formação docente e dos avaliadores/observadores. 
The systematic review also allow the identification of good practices in the use of programming environment for people who are blind, as well as the use of robots for these people. Among these, there are issues related to the use of resources and educational procedures, dynamic work with the students, forms of assessment of learning and the development environment, besides issues related to the location in which the activities will occur and the training of teachers and evaluators/observers.
% ESTE PARAGRAFO ESTA CONFUSO E NAO DA P ENTENDER O OBJETIVO DELE ... EU LEIO ASSIM ""TEM UM MONTE DE $%$#%#$ P RESOLVER"
% ===

% The Use of Robotics to Promote Computing to Pre-College Students with Visual Impairments

% Os autores justificam o uso de Lego ao invés de outros robôs pelo custo acessível, grande quantidade de material didático e a possibilidade de realizar trabalhos em casa.
% O ambiente de desenvolvimento é o BricxCC com NXC. Entretanto, o BricxCC possi algumas limitações como (ver tabela 1, pg 7):
% - dificuldade de leitura de pontuação da linguagem de programacao (paranteses, chaves, ponto-vírgula)
% - não permite a leitura da linha de código
% - não há áudio caso o robô perca comunicação com o computador
% - dificuldade de ler código de erro do compilador
% - uso de ícones sem recursos de acessabilidade
% - screen reader não funciona com 'virtual brick', que é uma espécie de robô simulado.

% ====

% An Accessible Robotics Programming Environment for Visually Impaired Users

% Devido aos problemas mencionados no artigo anteriror como BricxCC, os mesmos autores criaram um novo ambiente de programação chamado JBrick.

% "The default programming software available from Lego uses icons to represent commands. This software is not accessible to visually impaired users, most notably in terms of screen reader compatibility."

% "The JBrick user interface is designed to be accessible to programmers with various degrees of vision. In addition to compatibility with screen readers, refreshable braille displays, and magnification software, the user interface itself is designed to accommodate both sighted and low vision users."

% "Of particular issue were nested if/then statements. Impacting the issue was that some participants were not familiar with the use of punctuation such as braces and brackets, including their location on the keyboard."

% A linguagem usada chama-se ncx (Not eXactly C)

% ====

% Using Haptic and Auditory Interaction Tools to Engage Students with Visual Impairments in Robot Programming Activities

% Estes autores referenciam o 1o trabalho acima e seguem utilizando o mesmo material (Lego + BricxCC), mais incluiram um wiimote para feedback tátil e um agente chamado Robbie para narrar as mensagens de erro de compilação.

% Este artigo não discute muito as dificuldades encontradas com o BricxCC, mas o agente Robbie foi justamente criado por uma limitação do BricxCC, que não foi construído com o propósito de acessibilidade. Ou seja, dava erro de compilação e o usuário não conseguia identificar o erro e em que linha está o erro.

% ====