\begin{landscape}
\appendices
\section{Lista de comandos do GoDonnie}

% Imprime uma página indicando o início dos apêndices
% ----------------------------------------------------------

\begin{small}
\begin{longtable}{ p{3cm} p{3cm} p{8cm} p{8cm}}
\\
\label{tab:donnie-cmd}\\
\toprule
    \textbf{Comando} 
    & \textbf{Argumentos}  
    & \textbf{Explicação}  
    &\textbf{Exemplo}
\\ \hline
    \textbf{PF n}
    & n é o número de passos
    & Anda n passos para frente
    & PF 5 
    \newline

O robô andará 5 passos para frente. Supondo que o robô está na posição 0, 0 e virado para o norte, o comando PF 5 colocará o robô na posição 5, 0, mantendo a direção para o norte. 
\\ \hline
    \textbf{PT n}
    & n é o número de passos  
    & Anda n passos para trás. É como se andasse de ré.  
    & PT 5 
    \newline
    
    O robô andará 5 passos para trás. Supondo que o robô está na posição 5, 0 e virado para o norte, o comando PT 5 colocará o robô na posição 0, 0, mantendo a direção para o norte.  
\\ \hline
    \textbf{GD x}
    & x é número de graus 
    & Gira x graus para direita. Não há deslocamento do robô. 
    & GD 90
    \newline

    O robô irá girar 90 graus para direita. Supondo que o robô está virado para o norte, o comando GD 90 irá girar o robô 90 graus para a direita, mantendo-o na  direção leste.  
\\ \hline
    \textbf{GE x} 
    & x é número de graus 
    & Gira x graus para esquerda. Não há deslocamento do robô. 
    & GE 90
\newline

    O robô irá girar 90 graus para esquerda. Supondo que o robô está virado para o leste, o comando GE 90 irá girar o robô 90 graus para a esquerda, mantendo-o na  direção norte. 
\\ \hline
    \textbf{ESPIAR}
    & nenhum 
    & Retorna a identificação do objeto, um ângulo aproximado e a distância aproximada de colisão entre o robô e o objeto identificado. 
    \newline 
    O rastreamento para identificação dos objetos ocorre a 90º graus a esquerda e a direita da frente do robô. 
    & Supondo que o robô está na posição 2,3, virado para o norte, e que há um obstáculo verde na posição 0,5 e outro obstáculo vermelho na posição 6,3.
    \newline

    ESPIAR
    Resposta: verde, 40º a esquerda, 2 passos, vermelho, 90º a direita, 4 passos 
\\ \hline
    \textbf{DISTÂNCIA d}
    & d é a direção (f - frontal; fd - frontal direita; d - direita;  td - traseiro direito; t - traseiro; te - traseiro esquerda; e - esquerda; fe - frontal esquerda) 
    & Retorna a quantidade de passos do sensor do robô até um obstáculo, de acordo com a direção escolhida.
    \newline

    Distância F retorna o número de passos do robô até um objeto que foi detectado pelo sensor da parte da frente do robô. 
    \newline

    Distância FD retorna o número de passos do robô até um objeto que foi detectado pelo sensor da parte da frente lateral direita do robô. 
    \newline

    Distância TD retorna o número de passos do robô até um objeto que foi detectado pelo sensor da parte da trás lateral direita do robô.
    \newline

    Distância T retorna o número de passos do robô até um objeto que foi detectado pelo sensor da parte da traseira do robô. E, assim, sucessivamente.
    \newline

    Não havendo obstáculos, retorna a quantidade de passos que o sensor consegue identificar.
    &  
    DISTÂNCIA F
    DISTÂNCIA FD
    DISTÂNCIA FE
    DISTÂNCIA T
    DISTÂNCIA TE
    DISTÂNCIA TD
    \newline
    
    Supondo que o robô está na posição 0,0, virado para o norte e há obstáculos nas seguintes posições, o resultado será:
    \newline

    Obstáculo em 0, 3: 
    DISTÂNCIA F

    Resposta: 3 passos
\\ \hline
    \textbf{ESTADO}
    & nenhum & Informa a posição x e Y do robô no eixo cartesiano e último comando digitado, antes de Estado. 
    & PF 3 ESTADO
    \newline

    Supondo que o robô estava em 0,0. O robô andará 3 passos para frente e informará 0 3 PF 3.
    \newline

    Não havendo obstáculos, converter a distância máxima (3 metros) em passos. 
\\ \hline
    \textbf{CRIAR x}
    & x é uma variável que será criada 
    & Cria uma variável igual a um valor ou a uma expressão. Lembrar que ao criar uma variável novamente, ela irá sobrescrever o valor anterior. 
    & Exemplo 1: CRIAR A
    Cria uma variável chamada A.
    \newline

    A = 2
    Tendo sido criada a variável, pode atribuir um valor diretamente. A variável chamada A armazena 2.
    \newline

    CRIAR B =5
    Cria uma variável chamada B, que armazena o valor 5
    \newline

    CRIAR C = A + B
    Cria uma variável chamada C, que recebe o valor da variável A somado ao valor da variável chamada B. O resultado da variável C é 7.
    \newline
    
    C = 1
    Altera o valor da variável C e armazena o valor 1, perdendo o valor anterior.
    \newline

    D = 5
    Retornará erro porque a variável D ainda não foi criada. 
\\ \hline

    \textbf{POS k} 
    & k é um eixo do plano cartesiano (X ou Y) ou ângulo (A). 
    & Retorna a posição atual do robô no eixo X ou no eixo Y ou o ângulo atual do robô. & Supondo que o robô está na posição 0,0 virado para o norte:
    \newline
    
    PF 2

    GE 90

    PF 5

    POS X (retornará -5)

    POS Y (retornará 2)

    POS A (retornará 270)

    GD 90

    POS A (retornará 0) 

    GD 90 

    POS A (retornará 90)
\\ \hline
    \textbf{PARA inicialização; expressão operador lógico expressão; incremento ou decremento 
    FAÇA comandos FIM PARA}
    & Inicialização: variável  = algum valor inteiro

    variável ou Expressão operador lógico variável ou expressão:
    variável ou expressão - operador lógico - variável ou expressão
    
    Incremento: variável + constante ou variável + variável
    
    Decremento: variável - constante ou variável - variável & Repete os comandos enquanto  a Expressão-operador lógico-expressão for verdadeira. & Supondo que o robô está na posição 0,0, virado para o norte e precise fazer um quadrado de tamanho de lado 3.
    \newline

    PARA 
    
    a=1; a<=4; a=a+1 
    
    FAÇA  
    
    PF 3 GD 90 
    
    FIM PARA
    \newline

    Supondo que o robô comece na posição 0,0. A variável a começará com o valor 1 e repetirá os comandos PF 3 e GD 90 enquanto a variável a for menor ou igual a 4. Ao final, o robô terá feito um trajeto similar a um quadrado e finalizará na posição 0,0 virado para o norte. 
\\ \hline
    \textbf{REPITA n VEZES comandos FIM REPITA}
    & n é o número de vezes que os comandos serão repetidos.
    & Repete os comandos n vezes. 
    & 
    REPITA 4 VEZES 
    
    GD 90 
    
    PF 2 
    
    FIM REPITA
    \newline
    
    Supondo que o robô comece na posição 0,0. Os comandos PF 3  GD 90 serão repetidos 4 vezes. Ao final, o robô terá feito um trajeto similar a um quadrado e finalizará na posição 0,0 virado para o norte.
\\ \hline
    \textbf{SE expressão operador lógico expressão 
    ENTÃO comandos SENÃO comandos FIM SE}& expressão = variável ou expressão. 
    & Testa se uma condição é verdadeira e, em caso afirmativo, executa os primeiros comandos. Caso contrário, executa os comandos da expressão Senão. 
    & Supondo que, se a variável a for menor do que 4 o robô tenha que andar para frente 5 passos e caso contrário tenha que girar 45 graus para esquerda:
    \newline
    
    SE a\textless 4 
    
    ENTÃO PF 5 
    
    SENÃO GE 45
    
    FIM SE  
\\ \hline
    \textbf{BIP n, d}
    & n é a nota musical e d é a duração em segundos (do, ré, mi, fá, sol, lá, si) 
    & Faz o robô emitir um som de acordo com a nota musical que durará d segundos. Este som é emitido pelo robô ou pelo ambiente virtual, dependendo de quem estará ativo. 
    & 
    BIP do, 3
    
    BIP ré, 1
    
    BIP mi, 6
    
    BIP fá, 7
    
    BIP sol, 3
    
    BIP lá, 4
    
    BIP si, 9   
\\ \hline
    \textbf{SOM a}
    & a é o estado do áudio, que pode estar ligado ou desligado. 
    & Comando que liga ou desliga o áudio do recurso que estiver ativo, que poderá ser o robô ou o ambiente virtual. 
    & SOM LIGAR

    SOM DESLIGAR 
\\ \hline
    \textbf{APRENDER nome: variável1, variável2, variável3, … INICIO comandos FIM APRENDER}
    & nome é o nome da função e variavel1, variavel2, variavel3  são os argumentos da mesma 
    & É utilizado para “ensinar” comandos ao robô. &  O robô precisa executar vários retângulos de tamanhos diferentes. Para isso, pode ser criado um procedimento chamado RETÂNGULO que teria duas variáveis, uma para receber o tamanho da altura e outro o tamanho da base.  
    \newline   
       
    APRENDER RETÂNGULO: base, altura
    
    FAÇA
    
    PF base GD 90 
    
    PF altura
    
    GD 90 
    
    PF base 
    
    GD 90 
    
    PF altura 
    
    GD 90 
    
    FIM APRENDER
    \newline
    
    retângulo 5,3
    
    retângulo 8,4
    
    retângulo 9,5  
\\ \hline
    \textbf{FALAR x}
    &  x é uma palavra ou frase
    & Fala a palavra ou frase contida em x.  Este som é emitido pelo robô ou pelo ambiente virtual, dependendo de quem estará ativo.
    & FALAR “oi”
\\ \hline
    \textbf{ESPERAR x}
    & x é o tempo em segundos 
    & Espera x segundos para executar o próximo comando. 
    & Se o robô deve andar para frente 2 passos, esperar 3 segundos e andar mais 4 passos:
    \newline
        
    PF 2 
    
    ESPERAR 3
    
    PF 4 
\\ \hline
    \textbf{SAIR}
    & nenhum
    & Fecha o ambiente de programação. Só pode ser usado no terminal. 
    & SAIR
\\ \hline
    \textbf{HISTÓRICO}
    & nenhum
    & Lê o que está na tela. É possível navegar usando P, para ir para a próxima linha, A, para ir para a linha Anterior, ou pode pular para uma linha determinada. Para sair do histórico usa-se a tecla ESC.
    & Supondo que tenham sido escritos os seguintes comandos:
    \newline
        
    PF 3
    
    GD 45
    
    PF 6
    
    GD 45
    
    PF 3
    \newline
    
    HISTÓRICO
    
    “O programa tem 5 linhas. Você está na linha 6. Digite o número da linha desejada ou P para ir para a próxima linha ou A para ir para a linha Anterior ou ESC para sair do histórico".
    \newline
    
    P
    “Não existe comandos”
    
    A
    
    será falado PF 3
    
    A
    
    será falado GD 45
    
    1
    
    será falado PF 3
    
    ESC
    
    sairá do histórico
\\ \hline
    \textbf{CORES c}
    & c é a cor desejada (azul; vermelho; verde)
    & Verifica quantos objetos de determinada cor o robô consegue identificar na sua frente.
    & Supondo que há 1 objeto vermelho e 2 azuis
    \newline
    
    CORES vermelho
    
    será falado 1
    
    CORES azul
    
    será falado 2
\\ \hline
    \textbf{Operadores lógicos}
    &  
    + soma
    
    - subtração
    
    * multiplicação
    
    / divisão
    
    \textless\textgreater diferente
    
    == igual comparação
    
    = atribuição
    
    \textless menor
    
    \textgreater maior
    
    \textless= menor ou igual
    
    \textgreater= maior ou igual

    &  Operadores para comparar valores ou expressões.
    & 
\\ 
\bottomrule
\end{longtable}
\end{small}

\end{landscape}
