
\section{Conclusion and Future Work}
\label{sec:conclusion}

This paper presented a robotic programing environment, built with open hardware and software, that has been designed for people with normal vision and also for visual impaired people. The paper details the hardware, the firmware, the driver design, and the virtual simulation environment. Currently there is a Logo interpreter with sound feedback (text synthesis and event sounds) but it still requires additional design effort to integrate it with a screen reader and a magnification tool. Finally, we believe that the overall programing environment is almost complete and we are planning to start testing and evaluating it.

In the near future, we plan to evaluate the sound feedback for Stage with visual impairment users. It is unclear how effective this virtual environment will help to learn programming and if it is able to hold the student`s attention and curiosity for a long period. For this reason, we also plan to compare the system with virtual robots or the system with physical robots to check each one is more effective for the learning processes.

We also plan to evaluate if the environment helps the students to improve their O\&M and programming skills. 
%Para isso, está prevista a realização de testes piloto para validar a metodologia de uso da linguagem de programação e do uso do robô e para revisar o protocolo de avaliação de qualidade de uso do ambiente. 
We are planning test trials to validate the usage metodology of GoDonnie languague, the Donnie robot, and the quality assesment of the overall framework. 
