\section{Usage Scenarios}
\label{sec:usage}


%	Explicar como as coisas funcionam… Integração…
%	Falar da detecção de cores…. (associar objetos a cor)
%	Falar do uso da aplicação com o “cercadinho” (pensar em um cenário de uso)
	
	
This section describes the assistive capabilities for users with visual disabilities. 

Starting with the software, this module allows a navigation free or based in scenarios. In free navigation, the user interacts only with the terminal and commands a virtual robot. All commands are perceived by the student via the computer's sound interface. The scenario-based navigation uses the robot and allows the user to send commands from the terminal to the robot, which will move into a space.
%Na navegação livre, o usuário interage somente com o terminal e comanda um robo virtual. Todos os comandos são transmitidos pela interface sonora. A navegação baseada em cenário utiliza o robô e permite que o usuário envie comandos pelo terminal ao robo, que se deslocará em um espaço. 

The hardware part of the system enables the robot to generate tactile and auditory feedback for students with visual disabilities. The speaker connected to the RPi board enables the robot to speak using Festival Google TTS text-to-speech software and to generate pre-configured sounds to indicate, for instance, collision, obstacle position/distance, robot steps, turns, object recognition, goal achievement, among other events. The buzzer and vibrating motors are used for more important alerts such as robot collision and low battery, which require immediate attention from the student.	

Stage enables the creation, for example, of a virtual arena where all student`s robots can be together by connecting to the same sever. This can be used to create competition where the robots must find an object in the arena, the robots have to collect objects (e.g. gold gems) from the environment, or the robots can interact with each other for cooperation and building teams.

%\pessoa vidente é aquele que enxerga. Utilizamos os termos person who is not blind ou person who is not visually impaired. Substitui as expressões "who is a seer" para "who is not blind".... Marcia

Some examples of usage scenarios are presented below:

\begin{itemize}
    \item Elias, who is not blind, is the father of Luiza, who is blind. Luiza begin their studies at a new school. So Elias uses the robot to teach the school space for Luiza. For this, he explains the basic commands of the robot and explains the school space. So asks to Luiza what the robot should do to get from one place to another in the school.
    % Elias, que é vidente, é pai de Luiza, que é cega. Luiza começará seus estudos em uma nova escola. Então, Elias utiliza o robô para ensinar o espaço da escola para Luiza. Para isso, ele explica os comandos básicos do robô e explica o espaço da escola. Então, pede para que a Luiza diga o que o robô deve fazer para ir de um lugar a outro da escola.
    
    \item John, who is blind, is typing commands while Luis, who is not blind, observes the robot running  in the real scenario. John is having trouble to take the robot to its destination because he is attentive to the lines of code that he is typing, forgetting the global context. Luis, who observes and monitors the movement of the robot, said that the robot is having trouble to reach the destination. John and Luis discuss how to solve the problem and to make the robot reaching the goal. For this, John selects the editor option that provides information about the setting and context where the robot is while Luis tells you what he is watching.
    % João, que é cego, está digitando os comandos enquanto Luis, que é vidente, observa a execução do robô no cenário real. João está com dificuldades para levar o robô ao seu destino porque está atento às linhas de código que está digitando, esquecendo-se do contexto global. Luis, que observa e acompanha a movimentação do robô, informa que o robô está com dificuldades para chegar ao destino. João e Luis discutem sobre como solucionar o problema e fazer com que o robô alcance o objetivo. Para isso, João seleciona a opção do editor que fornece informações sobre o contexto do cenário e onde se encontra o robô enquanto o Luis informa o que está observando.
    
    \item Amanda, who is not blind and beginner in programming, decides to program the robot. Mary, who is blind and already have programming knowledge, will help her. For this, Maria proposes a path to be executed by the robot and explains the basic commands for Amanda. After, Amanda begins to write the program while Mary accompanies with touch, the movement of the robot in the scenario and provides feedback to Amanda. After the robot reaches the goal, Amanda and Mary review the commands and the robot trajectory and discuss if they could optimize the schedule.
    % Amanda, que é vidente e iniciante em programação, decide programar o robô. Maria, que é cega e já tem conhecimento de programação, irá ajudá-la. Para isso, Maria propõe um trajeto para ser executado pelo robô e explica os comandos básicos para a Amanda. Após, Amanda começa a escrever o programa enquanto Maria acompanha, com o tato, a movimentação do robô no cenário e fornece feedback para a Amanda. Após o robô alcançar o objetivo, Amanda e Maria revisam os comandos e a trajetória do robô e discutem se poderiam otimizar a programação.
    
    \item Afonso and Anna, who are blind, will program the robot together. They use touch to recognize the scenario, the initial position of the robot, and the target position. They discuss and decide what commands they should use. Later, Afonso enters each command while Ana verifies if the result is the expected, according to the previously agreed path. Ana verifies that the robot did not do a desired path and asks Afonso to read the commands previously typed while she simulates the robot's movement with touch. Together they check the wrong program segment and restart the trajectory from the starting position.
    % Afonso e Ana, que são cegos, vão programar juntos o robô. Para isso, utilizando o tato, reconhecem o cenário, a posição inicial do robô e a posição de destino. Discutem e decidem que comandos vão utilizar. Após, o Afonso vai digitando cada comando enquanto a Ana verifica se o resultado é o esperado, de acordo com a trajetória previamente combinada. Ana verifica que o robô não fez um caminho desejado e pede para que o Afonso leia os comandos já digitados enquanto ela simula com o tato o movimento do robô. Em conjunto, verificam o trecho de programa que está errado e refazem a trajetória começando da posição inicial.
    

    \item Professor Gabriel teaches programming classes for a class that has sighted and blind students. He proposes an activity of program optimization. For this, he has developed a program which the robot goes from a starting point to the destination making several turns and passing close to 5 objects. The challenge to the students is to move the robot from the starting point to the destination using the shortest way, and passing close to only two objects. All students have explored the scenario only with the touch. So the seers students were blindfolded. Afterwards, the students discussed and built models to use as support for programming. Professor Gabriel explained their program and asked that the students, in groups of three students, rewrite the program. After the timeout, all programs were discussed and the group has chosen the best resolution.
    %O professor Gabriel ministra aulas de programação para uma turma que possui alunos videntes e alunos cegos. Ele propôs uma atividade de otimização de programa. Para isso, elaborou um programa que fazia o robô partir de um ponto inicial e chegar a um destino fazendo várias voltas e passando próximo a 5 objetos. O desafio imposto aos alunos foi de que fizessem o robô sair do ponto inicial e alcançasse o ponto final utilizando o menor trajeto, e passando próximo a somente 2 objetos. Todos os alunos exploraram o cenário somente com o tato. Por isso, os alunos videntes ficaram com os olhos vendados. Após, os alunos discutiram e construiram maquetes para utilizar como apoio à programação. O professor Gabriel explicou seu programa e pediu para que os alunos, em grupos de 3 alunos, reescrevessem o programa. Passado o tempo limite, todos os programas foram discutidos e a turma elegeu a melhor resolução.
    
    \item Ricardo, who is not blind, and Denise, who is blind, need to build a scenario of a particular museum. For this, they explore the virtual environment, they discuss and tell to the others where the objects that will represent the museum's experiments on the scene should be placed.
    % Ricardo, que é vidente, e Denise, que é cega, precisam construir um cenário de um determinado museu. Para isso, exploram o ambiente virtual, discutem e vão informando aos demais aonde devem ser colocados os objetos que irão representar os experimentos do museu no cenário.

\end{itemize}